\documentclass[12pt, a4paper]{article}
\usepackage{geometry}
\usepackage{multicol}
\usepackage{multirow}
\usepackage[table]{xcolor}
\usepackage{enumerate}
\usepackage{listings}
\usepackage{minted}
\geometry{margin=1in}

\usepackage[utf8]{inputenc}
\usepackage[T1]{fontenc}
\usepackage{graphicx}
\usepackage{ragged2e}

\usepackage{minted}
\usepackage{rotating}
\usepackage{hyperref}
\usepackage{float}
\usepackage{pstricks}
\usepackage{makecell}
\usepackage{fancyhdr}
\pagestyle{fancy}
\fancyhead{}
\fancyfoot{}

\fancyhead[l]{Laboratorio de Computación Gráfica e Interacción humano-computadora}
\fancyhead[r]{Grupo: 02}
\fancyfoot[r]{\thepage}

\usepackage{amsmath}
\usepackage{tikz}
\usepackage{longtable}
\usetikzlibrary{automata, positioning, arrows,shapes.misc, shapes.arrows, chains,matrix,positioning,% wg. " of "
	scopes,%
	decorations.pathmorphing,% /pgf/decoration/random steps | erste Graphik
	shadows}
\renewcommand{\refname}{Referencias}
\renewcommand{\contentsname}{Contenidos}
\renewcommand{\figurename}{Figura}
\renewcommand{\partname}{Parte}

\begin{document}
	
	\begin{titlepage}
		
		\newcommand{\HRule}{\rule{\linewidth}{0.5 mm}} % Define comando para lineas horizontales
		
		\centering % Centra todo en la pagina
		
			%----------------------------------------------------------------------------------------
		%   LOGO SECTION
		%----------------------------------------------------------------------------------------
		
		\includegraphics[scale = 0.235]{img/logo.png} % Logo
		\hspace{4cm}
		\includegraphics[scale = .32]{img/fi.png}\\[.65cm] % Logo
	%----------------------------------------------------------------------------------------
		%   ENCABEZADO
		%----------------------------------------------------------------------------------------
		
		\textsc{\large \bfseries UNIVERSIDAD NACIONAL AUTÓNOMA DE MÉXICO}\\[.5 cm] % Nombre de universidad
		\textsc{\large FACULTAD DE INGENIERÍA}\\[0.5 cm] % Division
		\textsc{\large Ingeniería en Computación}\\[1.4 cm] % Carrera
		
		%----------------------------------------------------------------------------------------
		%   TITLE SECTION
		%----------------------------------------------------------------------------------------
		{\large LABORATORIO DE COMPUTACIÓN GRÁFICA\\ E INTERACCIÓN HUMANO COMPUTADORA}\\[.4 cm] % Materia
		% llllllllllllllllllllllllllllllllllllll
		\textsc{\large Grupo: 02}\\[1.5 cm]
		\textsc{\large Ejercicio de clase}\\[0.5 cm]
		\textsc{\large Práctica Número 4}\\[1.5 cm]
		{\large Alumna: Pamela Salgado Fernández Pamela}\\[.3 cm]
		{\large Número de Cuenta: 313236505}\\[.3 cm]
		{\large Email: pame501@yahoo.com.mx}\\[1.6 cm]
		\raggedleft 
		{\large Semestre 2019-2}\\[.3 cm]
		\textsc{\large Grupo de teoría: 1}\\[.3 cm]
		{\large Fecha de entrega límite: 27/02/2019}\\[.5 cm] 


			%----------------------------------------------------------------------------------------
		
		\vfill % Llenar el resto de la página con espacio en blanco
		 
		
	\end{titlepage}
	\tableofcontents
	\newpage
	\noindent
\section{Actividades realizadas}
\justify
Para la actividad de clase, el profesor nos solicitó que a partir del código que nos proporcionó, lo modificáramos para crear 9 triángulos, es decir, instanciar varias veces la figura.
\vspace{.35cm}
Para la segunda actividad de clase, fue necesario asignar un color diferente a cada uno de los triángulos.
\justify
\subsection{Instanciar el triángulo varias veces}
Para poder crear varios triángulos solo fue necesario llamar de nuevo a model para inicializar de nuevo la matriz y una vez teniendo la matriz agregamos todos los datos necesarios para que nuestro otro triangulo se vea (escala,translación y rotación) y se los  . \\[.15cm]
En practica4.cpp dentro del while que se encuentra en la linea 151, se agregan los demás triángulos:

	\begin{minted}[breaklines]{C++}
//************************************************************************
//creamos los diferentes triangulos 
//HACIA ABAJO
//a model le paso una matriz identidad para resetear los datos 
model = glm::mat4(1.0);
model = glm::translate(model, glm::vec3(-0.15f, -0.3f, -1.0f));
model = glm::scale(model, glm::vec3(0.3f, 0.3f, 0.3f));
model = glm::rotate(model,  180* toRadians, glm::vec3(1.0f, 0.0f, 0.0f));
glUniformMatrix4fv(uniformModel, 1, GL_FALSE, glm::value_ptr(model));
meshList[1]->RenderMesh();

		//segundo Triangulo
//a model le paso una matriz identidad para resetear los datos 
model = glm::mat4(1.0);
model = glm::translate(model, glm::vec3(0.15f, -0.3f, -1.0f));
model = glm::scale(model, glm::vec3(0.3f, 0.3f, 0.3f));
model = glm::rotate(model, 180 * toRadians, glm::vec3(1.0f, 0.0f, 0.0f));
glUniformMatrix4fv(uniformModel, 1, GL_FALSE, glm::value_ptr(model));
meshList[2]->RenderMesh();

		// Tercer triangulo
//a model le paso una matriz identidad para resetear los datos 
model = glm::mat4(1.0);
model = glm::translate(model, glm::vec3(0.3f, -0.3f, -1.0f));
model = glm::scale(model, glm::vec3(0.3f, 0.3f, 0.3f));
glUniformMatrix4fv(uniformModel, 1, GL_FALSE, glm::value_ptr(model));
meshList[3]->RenderMesh();

//************************************************************************
	\end{minted}
	
\vspace{.1cm}	
\justify
Este mismo procedimiento se sigue para los otros triángulos.

\subsubsection{Ejecución del programa}
	\centering 
	\includegraphics[scale = .90]{img/01_p4.JPG}\\[.25cm] % Imagen1
	Figura 1 \\[.55cm]
	\vspace{0.4 cm}
	
	
\raggedright	
\subsection{Cambiar color a cada uno de los triángulos}
\justify
Para cambiar el color de los triángulos, fue necesario crear 9 VBOs, los cuales contenían el color que se asignaría a cada uno de los triángulos, para ello nos apoyamos también en una función llamada genera\_color().\\[.02cm]

	\begin{minted}[breaklines]{C++}
//************************************************************************

// creamos VBO's los cuales nos darán el color de cada vertice
	GLfloat color_triangulo[] = {
		c1, c2, c3, 1.0f,
		c1, c2, c3, 1.0f,
		c1, c2, c3, 1.0f,
		c1, c2, c3, 1.0f,
	};

	GLfloat color_triangulo1[] = {
	c2, c2, c1, 1.0f,
	c2, c2, c1, 1.0f,
	c2, c2, c1, 1.0f,
	c2, c2, c1, 1.0f,
	};
	genera_color();
	GLfloat color_triangulo2[] = {
	c2, c1, 0.0f, 1.0f,
	c2, c1, 0.0f, 1.0f,
	c2, c1, 0.0f, 1.0f,
	c2, c1, 0.0f, 1.0f,
	};
	genera_color();
	GLfloat color_triangulo3[] = {
	c4, 0, c1, 1.0f,
	c4, 0, c1, 1.0f,
	c4, 0,c1, 1.0f,
	c4, 0, c1, 1.0f,
	};

//************************************************************************
	\end{minted}
	
Esto se realiza para cada uno de los triángulos.\\
La función para cambiar de color es la siguiente:
\vspace{.25cm}

	\begin{minted}[breaklines]{C++}
	//**** para cambiar el color *******
float c1= 1.0f, c2=0.0f, c3=0.0, c4=0.8;
void genera_color() {
	c1 = 0;
	c1 = 0.3 + (float)rand() / RAND_MAX;
	c2 = 0;
	c2 = 0.6 + (float)rand() / RAND_MAX;
	c3 = 0;
	c3 = 0.5 + (float)rand() / RAND_MAX;
	c4 = 0;
	c4 = 0.6 + (float)rand() / RAND_MAX;
}
	\end{minted}
	
También se modificó mesh.cpp para que recibiera el VAO del color.


	\begin{minted}[breaklines]{C++}
void Mesh::CreateMesh(GLfloat *vertices, unsigned int *indices, unsigned int numOfVertices, unsigned int numberOfIndices, GLfloat *color, unsigned int tam_color)
{

	indexCount = numberOfIndices;
	glGenVertexArrays(1, &VAO); //generar 1 VAO
	glBindVertexArray(VAO);//asignar VAO

	glGenBuffers(1, &IBO);
	glBindBuffer(GL_ELEMENT_ARRAY_BUFFER, IBO);
	glBufferData(GL_ELEMENT_ARRAY_BUFFER, sizeof(indices[0]) * numberOfIndices, indices, GL_STATIC_DRAW);

	glGenBuffers(1, &VBO);
	glBindBuffer(GL_ARRAY_BUFFER, VBO);
	glBufferData(GL_ARRAY_BUFFER, sizeof(vertices[0]) * numOfVertices, vertices, GL_STATIC_DRAW);


	glVertexAttribPointer(0, 3, GL_FLOAT, GL_FALSE, 0, 0);
	glEnableVertexAttribArray(0);
	// VBO con el color
	glGenBuffers(1, &VBO);
	glBindBuffer(GL_ARRAY_BUFFER, VBO);
	glBufferData(GL_ARRAY_BUFFER, sizeof(color[0]) * tam_color, color, GL_STATIC_DRAW); 
	glVertexAttribPointer(1, 4, GL_FLOAT, GL_FALSE, 0, 0);
	glEnableVertexAttribArray(1);

	glBindBuffer(GL_ARRAY_BUFFER, 0);
	glBindBuffer(GL_ELEMENT_ARRAY_BUFFER, 0);
	glBindVertexArray(0);

}
	\end{minted}
	\subsubsection{ejecución del código}
		\centering 
	\includegraphics[scale = .90]{img/02_p4.JPG}\\[.25cm] % Imagen1
	Figura 2 \\[.55cm]
	\vspace{0.4 cm}
	
\raggedright 
\section{Código}
Anexo a este documento, se encuentran los códigos creados para realizar esta práctica, tanto como los archivos .cpp como los .h.\\

 \raggedright 	
\section{Problemas presentados}
\justify
No se presentaron problemas durante esta práctica. \\[.2cm]

\section{Conclusiones}
\justify
La realización de estos ejercicios no fue difícil, ya que utilizamos todo lo aprendido, lo único nuevo fue la creación de más objetos y la función ortho. \\[0.4cm]
La parte que costó más trabajo fue la realización de la pirámide con los 9 triángulos ya que teníamos que desplazarlos y rotarlos de tal forma que no hubieran espacios en blanco.

\end{document}